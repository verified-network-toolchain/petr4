\newcommand{\VVal}[1]{#1'}
\newcommand{\VVVal}[1]{#1''}

\newcommand{\A}{A}
\newcommand{\synCat}{\mathit{symbol}}

%rule names:
\newcommand{\rn}[1]{\TirName{#1}}
\newcommand{\rSep}{\rn{-}}
\newcommand*{\rE}{\rSep\rn{E}}
\newcommand*{\rAE}{\rSep\rn{AE}}
\newcommand*{\rT}{\rSep\rn{T}}
\newcommand*{\rS}{\rSep\rn{S}}
\newcommand*{\rAS}{\rSep\rn{AS}}
\newcommand*{\rD}{\rSep\rn{D}}
\newcommand*{\rAD}{\rSep\rn{AD}}

\newcommand*{\ruleName}{\rn{RuleName}}
\newcommand*{\ruleNameE}{\ruleName\rE}
\newcommand*{\ruleNameAE}{\ruleName\rAE}
\newcommand*{\ruleNameT}{\ruleName\rT}
\newcommand*{\ruleNameS}{\ruleName\rS}
\newcommand*{\ruleNameD}{\ruleName\rD}

\newcommand*{\enumWellT}{\rn{Enum}\rT}
\newcommand*{\serenumWellT}{\rn{SerializableEnum}\rT}
\newcommand*{\serenumWellAltT}{\rn{SerEnum}\rSep\rn{Alt}\rT}
\newcommand*{\serenumTypedefWellT}{\rn{SerializableEnum}\rSep\rn{Typedef}\rT}
\newcommand*{\serenumTypedefWellAltT}{\rn{SerEnum}\rSep\rn{Typedef}\rSep\rn{Alt}\rT}
\newcommand*{\headerTypWellT}{\rn{Header}\rT}
\newcommand*{\headerGenericTypWellT}{\rn{Generic}\rSep\rn{Header}\rT}
\newcommand*{\headerUnionTypWellT}{\rn{HeaderUnion}\rT}
\newcommand*{\structTypWellT}{\rn{Struct}\rT}
\newcommand*{\parserTypWellT}{\rn{Parser}\rT}
\newcommand*{\controlTypWellT}{\rn{Control}\rT}
\newcommand*{\packageTypWellT}{\rn{Package}\rT}

\newcommand*{\boolT}{\rn{Bool}\rT}
\newcommand*{\stringT}{\rn{String}\rT}
\newcommand*{\intT}{\rn{(InfinitPrecision)Integer}\rT}
\newcommand*{\varbitT}{\rn{Varbit}\rSep\rT}
\newcommand*{\varbitExpT}{\rn{Varbit}\rSep\rn{Exp}\rT}
\newcommand*{\errT}{\rn{Error}\rT}
\newcommand*{\voidT}{\rn{Void}\rT}
\newcommand*{\dontcareT}{\rn{Dontcare}\rT}
\newcommand*{\notInstantiatedT}{\rn{NotInstantiated}\rT}
\newcommand*{\matchkindT}{\rn{MatchKind}\rT}
\newcommand*{\enumOneT}{\rn{Enum}\rT}
\newcommand*{\intWidthExpT}{\rn{SignedInt}\rSep\rn{Exp}\rT}
\newcommand*{\bitWidthExpT}{\rn{BitString}\rSep\rn{Exp}\rT}
\newcommand*{\intWidthT}{\rn{SignedInt}\rT}
\newcommand*{\bitWidthT}{\rn{BitString}\rT}
\newcommand*{\arrayT}{\rn{Array}\rT}
\newcommand*{\arraySpecT}{\rn{Array}\rSep\rn{Specialized}\rT}
\newcommand*{\tupleT}{\rn{Tuple}\rT}
\newcommand*{\setT}{\rn{Set}\rT}
\newcommand*{\enumTwoT}{\rn{SerializableEnum}\rT}
\newcommand*{\recordT}{\rn{Record}\rT}
\newcommand*{\headerUnionT}{\rn{HeaderUnion}\rT}
\newcommand*{\headerT}{\rn{Header}\rT}
\newcommand*{\structT}{\rn{Struct}\rT}
\newcommand*{\newTypeT}{\rn{NewType}\rT}
%% \newcommand*{\specializedExternT}{\rn{SpecializedType}\rSep\rn{ExternBase}\rT}
\newcommand*{\specializedT}{\rn{SpecializedType}\rT}
\newcommand*{\packageT}{\rn{Package}\rT}
\newcommand*{\controlT}{\rn{Control}\rT}
\newcommand*{\parserT}{\rn{Parser}\rT}
\newcommand*{\externT}{\rn{Extern}\rT}
\newcommand*{\functionT}{\rn{Function}\rT}
\newcommand*{\actionT}{\rn{Action}\rT}
\newcommand*{\constructorT}{\rn{Constructor}\rT}
\newcommand*{\tableT}{\rn{Table}\rT}
\newcommand*{\typeNameT}{\rn{Reference}\rT}
\newcommand*{\listTeq}{\rn{Lists}\rT}
\newcommand*{\tupleTeq}{\rn{Tuples}\rT}
\newcommand*{\headerTeq}{\rn{Headers}\rT}
\newcommand*{\headerUnionTeq}{\rn{HeaderUnions}\rT}
\newcommand*{\structTeq}{\rn{Structs}\rT}
\newcommand*{\enumTeq}{\rn{Enums}\rT}
\newcommand*{\serEnumTeq}{\rn{SerializableEnums}\rT}
\newcommand*{\setTeq}{\rn{Sets}\rT}

\newcommand*{\boolE}{\rn{Bool}\rE}
\newcommand*{\stringE}{\rn{String}\rE}
\newcommand*{\integerE}{\rn{Integer}\rE}
\newcommand*{\bitStringE}{\rn{FixedLengthInt}\rE}
\newcommand*{\signedIntE}{\rn{SignedInt}\rE}
\newcommand*{\nameE}{\rn{Name}\rE}
\newcommand*{\arrayAccessE}{\rn{ArrayAccess}\rE}
\newcommand{\headerStackIndexE}{\rn{HeaderStackIndex}\rE}
%% \newcommand{\headerStackIndexE}{\rn{HeaderStackIndex}\rSep\rn{bit}\rE}
%% \newcommand{\headerStackIndexE}{\rn{HeaderStackIndex}\rSep\rn{int}\rE}
%% \newcommand{\headerStackIndexE}{\rn{HeaderStackIndex}\\rE}
\newcommand*{\tupleAccessE}{\rn{TupleAccess}\rE}
\newcommand*{\bitSliceE}{\rn{UnsignedIntSlice}\rE}
\newcommand*{\intSliceE}{\rn{SignedIntSlice}\rE}
\newcommand*{\sliceWithCastE}{\rn{Slice}\rSep\rn{InsertImplicitCast}\rE}
\newcommand*{\listE}{\rn{List}\rE}
%% \newcommand*{\listAltE}{\rn{List}\rSep\rn{Alt}\rE}
\newcommand*{\recordE}{\rn{Record}\rE}
\newcommand*{\logicalNegE}{\rn{LogicalNegation}\rE}
\newcommand*{\bitwiseComplementE}{\rn{BitwiseComplement}\rE}
\newcommand*{\unaryMinusE}{\rn{UnaryMinus}\rE}
\newcommand*{\binaryOpsE}{\rn{BinaryOps}\rE}
\newcommand*{\castE}{\rn{Cast}\rE}
\newcommand*{\castTypNameE}{\rn{Cast}\rSep\rn{TypeName}\rE}
\newcommand*{\typeMemE}{\rn{TypeMember}\rE}
\newcommand*{\errMemE}{\rn{ErrorMember}\rE}
%% \newcommand*{\expMemE}{\rn{ExpressionMember}\rE}
\newcommand*{\structMemE}{\rn{StructMember}\rE}
\newcommand*{\sizesMemStructLikeE}{\rn{StructLike}\rSep\rn{SizeMember}\rE}
\newcommand*{\sizesMemSerTypesE}{\rn{SerializableTypes}\rSep\rn{SizeMember}\rE}
\newcommand*{\isValidMemE}{\rn{IsValidMember}\rE}
\newcommand*{\setValidMemE}{\rn{SetValidMember}\rE}
\newcommand*{\sizeMemHeaderStackE}{\rn{HeaderStack}\rSep\rn{SizeMember}\rE}
\newcommand*{\nextIndexMemHeaderStackE}{\rn{NextIndexMember}\rE}
\newcommand*{\nextLastMemHeaderStackE}{\rn{NextOrLastMember}\rE}
\newcommand*{\lastIndexMemHeaderStackE}{\rn{LastIndexMember}\rE}
\newcommand*{\pushPopMemHeaderStackE}{\rn{PushOrPopMember}\rE}
\newcommand*{\ternaryE}{\rn{Conditional}\rE} 
%% \newcommand*{\ternaryIntE}{\rn{Conditional}\rSep\rn{Int}\rE}
\newcommand*{\funcCallE}{\rn{FunctionCall}\rE\rSep\rn{1}}
\newcommand*{\funcCallNoTypeArgE}{\rn{FunctionCall}\rE\rSep\rn{2}}
\newcommand*{\instE}{\rn{Instantiation}\rE}
\newcommand*{\maskE}{\rn{Mask}\rE}
\newcommand*{\rangeE}{\rn{Range}\rE}

\newcommand*{\shiftRAE}{\rn{ShiftRight}\rAE}
\newcommand*{\shiftLAE}{\rn{ShiftLeft}\rAE}
\newcommand*{\bitConcatAE}{\rn{BitConcatenation}\rAE}
\newcommand*{\restAE}{\rn{Rest}\rAE}


%function call helper
\newcommand*{\nameFuncNameE}{\rn{Name}\rSep\rn{Func}\rAE\rSep\rn{1}}
\newcommand*{\nameFuncCountE}{\rn{Name}\rSep\rn{Func}\rAE\rSep\rn{2}}
\newcommand*{\nameActionE}{\rn{Name}\rSep\rn{Action}\rAE}
%% \newcommand*{\nameActionNameE}{\rn{NameExpression}\rSep\rn{ActionType}\rSep\rn{ResolveByName}\rE}
%% \newcommand*{\nameActionCountE}{\rn{NameExpression}\rSep\rn{ActionType}\rSep\rn{ResolveByCount}\rE}
\newcommand*{\expMemExternNameE}{\rn{ExpMem}\rSep\rn{Extern}\rAE\rSep\rn{1}}
\newcommand*{\expMemExternCountE}{\rn{ExpMem}\rSep\rn{Extern}\rAE\rSep\rn{2}}
\newcommand*{\expMemSpcE}{\rn{ExpMem}\rSep\rn{Specialized}\rAE}
%% \newcommand*{\expMemSpcCountE}{\rn{Exp:ExpMem}\rSep\rn{Type:Specialized}\rSep\rn{Resolve:Count}\rE}

\newcommand*{\inferTypeArgAE}{\rn{InferTypeArg}\rAE}

\newcommand*{\assignmentS}{\rn{Assignment}\rS}
\newcommand*{\emptyS}{\rn{Empty}\rS}
\newcommand*{\blockS}{\rn{Block}\rS}
\newcommand*{\returnS}{\rn{Return(inApplyBlockOrAction)}\rS}
\newcommand*{\returnEmpS}{\rn{ReturnEmpty}\rS}
\newcommand*{\returnFuncS}{\rn{Return(inFunction)}\rS}
\newcommand*{\exitS}{\rn{Exit}\rS}
\newcommand*{\ifthenS}{\rn{IfThen}\rS}
\newcommand*{\ifthenelseS}{\rn{IfThenElse}\rS}
\newcommand*{\switchS}{\rn{Switch}\rS}
\newcommand*{\declS}{\rn{Declaration}\rS}

\newcommand*{\constantD}{\rn{Constant}\rD}
\newcommand*{\instantD}{\rn{Instantiation}\rD}
\newcommand*{\parserD}{\rn{Parser}\rD}
\newcommand*{\controlD}{\rn{Control}\rD}
\newcommand*{\funcD}{\rn{Function}\rD}
\newcommand*{\actionD}{\rn{Action}\rD}
\newcommand*{\externFuncD}{\rn{ExternFunction}\rD}
\newcommand*{\varD}{\rn{Variable}\rD}
\newcommand*{\varInitD}{\rn{VariableInit}\rD}
\newcommand*{\valueSetD}{\rn{ValueSet}\rD}
\newcommand*{\tableD}{\rn{Table}\rD}
\newcommand*{\headerD}{\rn{Header}\rD}
\newcommand*{\headerUnionD}{\rn{HeaderUnion}\rD}
\newcommand*{\structD}{\rn{Struct}\rD}
\newcommand*{\errD}{\rn{Error}\rD}
\newcommand*{\matchkindD}{\rn{MatchKind}\rD}
\newcommand*{\enumD}{\rn{Enum}\rD}
\newcommand*{\serEnumD}{\rn{SerializableEnum}\rD}
\newcommand*{\serEnumTypdefD}{\rn{SerializableEnum}\rSep\rn{Typedef}\rD}
\newcommand*{\externObjD}{\rn{ExternObject}\rD}
\newcommand*{\typdefD}{\rn{Typedef}\rD}
\newcommand*{\typdefDclD}{\rn{Typedef}\rSep\rn{Declared}\rD}
\newcommand*{\newtypeD}{\rn{NewType}\rD}
\newcommand*{\newtypeDclD}{\rn{NewType}\rSep\rn{Declared}\rD}
\newcommand*{\controlTypD}{\rn{ControlType}\rD}
\newcommand*{\parserTypD}{\rn{ParserType}\rD}
\newcommand*{\packageTypD}{\rn{PackageType}\rD}
\newcommand*{\externD}{\rn{Extern}\rD}

%stuff here and there:
%TODO: fix mathit in the followings:
\renewcommand{\or}{\ | \ }
\newcommand{\whereBulletIs}{,\textit{ where } \bullet \textit{ is }}
\newcommand{\whereIs}[1]{,\textit{ where } {#1} \textit{ is }}
\newcommand{\where}[1]{,\textit{ where } {#1}}
\newcommand{\textOr}{\textit{ or }}
\newcommand{\numeric}[1]{\mathit{numeric}({#1})}
\newcommand{\error}{\mathit{error}}
\newcommand{\cond}{\mathit{cond}}
\newcommand{\prim}[1]{{#1}^\prime}
\newcommand{\pprim}[1]{{#1}^{\prime\prime}}
\newcommand{\ppprim}[1]{{#1}^{\prime\prime\prime}}
\newcommand{\so}[2]{{#1} \Rightarrow {#2}}
\newcommand{\lookupVarEnv}[1]{\varEnv({#1})}
\newcommand{\lookupExternEnv}[1]{\externEnv({#1})}
\newcommand{\lookupTypEnv}[1]{\typEnv({#1})}
\newcommand{\lookupTypSynEnv}[1]{\typSynEnv({#1})}
\newcommand{\lookupEnv}[3]{{#1}.{#2}({#3})}
\newcommand{\emp}{[\ ]}
\newcommand{\updateEnv}[1]{\env \ \mathsf{with}\ {#1}}
\newcommand{\updateEnvWithEnv}[2]{{#1} \ \mathsf{with}\ {#2}}
\newcommand{\insertToVarEnv}[3]{\varEnv[{#1} : ({#2}, {#3}) ]}
\newcommand{\insertsToVarEnv}[3]{\varEnv[\overline{{#1} : ({#2}, {#3}) }]}
\newcommand{\insertToTypEnv}[2]{\typEnv[{#1} = {#2}]}
\newcommand{\insertToTypSynEnv}[2]{\typSynEnv[{#1} = {#2}]}
\newcommand{\insertToTypParEnv}[1]{\typParEnv[{#1}]}
\newcommand{\insertsToTypEnv}[2]{\typEnv[\overline{{#1} = {#2}}]}
\newcommand{\insertsToConstEnv}[2]{\constEnv[\overline{{#1} = {#2}}]}
\newcommand{\insertToConstEnv}[2]{\constEnv[{{#1} = {#2}}]}
\newcommand{\envWhere}[1]{\env \ \mathsf{where}\ {#1}}
\newcommand{\envWhereCtxt}[1]{\envWhere {\ctxt = {#1}}}
%% \newcommand{\insertToTypEnv}[1]{\typEnv[{#1} \ \mathsf{var}]}
%% \newcommand{\insertToTypsEnv}[1]{\typEnv[\overline{{#1} \ \mathsf{var}}]}
\newcommand{\insertToEnv}[3]{{#1}.{#2}[{#3}]}
\newcommand{\suchThat}[2]{{#1}\ . \ {#2}}
\newcommand{\tr}{\mathsf{true}}
\newcommand{\fl}{\mathsf{false}}

% renew these commands to change the size and indentation of program text
%% \newcommand{\progind}{\parindent}
%% \newcommand{\progfontsize}{\small}
%% \newcommand{\progformat}{}
%% % in-line code
%% \newcommand{\prog}[1]{\textrm{\progfontsize\progformat\texttt{#1}}}
%% \newcommand{\tr}{\prog{true}}
%% \newcommand{\fl}{\prog{false}}

%% \newcommand{\iLeqLeq}[2]{{#1} \leq i \leq {#2}. \ } %TODO: refactor 1 < i < n
%% \newcommand{\iLeqL}[2]{{#1} \leq i < {#2}. \ }
%% \newcommand{\iEqLeq}[2]{{#1} < i \leq {#2}. \ }
%% \newcommand{\iLL}[2]{{#1} < i < {#2}. \ }
%% \newcommand{\leqLeq}[3]{{#1} \leq {#2} \leq {#3}. \ }
%% \newcommand{\leqL}[3]{{#1} \leq {#2} < {#3}. \ }
%% \newcommand{\lLeq}[3]{{#1} < {#2} \leq {#3}. \ }
%% \newcommand{\lL}[3]{{#1} < {#2} < {#3}. \ }
%% \newcommand{\forallcond}[3]{\forall {#1} \in {#2}\ . \ {#3}}

%typing judgments:
%checker
\newcommand{\expCheck}[2]{\env \vdash {#1} : {#2}}
\newcommand{\expCheckEnv}[3]{{#1} \vdash {#2} : {#3}}
\newcommand{\prmCheck}[2]{{#1} \vdash {#2}}
\newcommand{\dclCheck}[2]{\env \vdash {#1} \dashv {#2}}
\newcommand{\allcastenv}[2]{\env \vdash {#1} \rightarrow {#2}}
%inference
\newcommand{\envOne}[2]{\Delta, T, \Gamma \vdash {#1} : {#2}}
\newcommand{\expenv}[4]{\constEnv, \varEnv, \typEnv, \ctxt \vdash {#1} \leadsto {#2}, {#3}, {#4}}
\newcommand{\expenvv}[5]{\constEnv_{#5}, \varEnv_{#5}, \typEnv_{#5}, \ctxt_{#5} \vdash {#1} \leadsto {#2}, {#3}, {#4}}
\newcommand{\expenvvv}[3]{{#1}, \ctxt \vdash {#2} \leadsto {#3}} %%TODO: check where you've used this and correct the envs
\newcommand{\expenvWithCtxt}[5]{\constEnv, \varEnv, \typEnv, {#1} \vdash {#2} \leadsto {#3}, {#4}, {#5}}
\newcommand{\coerceBinArgsEnv}[3]{\env, \ctxt \vdash {#1} \twoheadrightarrow ({#2}), ({#3})}
%% \newcommand{\binOpEnv}[7]{\env \vdash {#1}, ({#4}, {#5}), ({#6}, {#7}) : {#2}, {#3}}
\newcommand{\typWellFormed}[1]{\env \vdash {#1}}
\newcommand{\typWellFormedNoEnv}[1]{\vdash {#1}}
\newcommand{\typNotWellFormed}[1]{\env \nvdash {#1}}
\newcommand{\typWellFormedUpdatedEnv}[2]{{#1}\vdash {#2}}
%% \newcommand{\typWellFormedWithEnv}[3]{{#1}, {#2}, \externEnv \vdash {#3}} %delete after fixing harim's inference rules.
\newcommand{\typEqEnv}[3]{ {#2} ==_{\env,{#1}} {#3}} %%todo: fix env
\newcommand{\typEq}[2]{{#1} ==_\env {#2}}
%context. type of expression. type of returned name.exp
\newcommand{\fieldAccessEnv}[4]{\env, {#1} \vdash {#2}, {#3} : {#4}}
\newcommand{\stmtenv}[3]{\env, \ctxt \vdash {#1} \leadsto {#2}, {#3} \dashv \env}
\newcommand{\stmtenvv}[5]{\env_{#1}, \ctxt_{#1} \vdash {#2} \leadsto {#3}, {#4} \dashv \env_{#5}}
%% \newcommand{\dclenv}[2]{\env, \ctxt \vdash {#1} \leadsto {#2} \dashv \env}
\newcommand{\dclenv}[6]{\varEnv, \constEnv, \typEnv, \typSynEnv \vdash {#1} \leadsto {#2} \dashv {#3}, {#4}, {#5}, {#6}}
%% \newcommand{\dclenvv}[4]{\env_{#1}, \ctxt_{#1} \vdash {#2} \leadsto {#3} \dashv \env_{#4}}
%% \newcommand{\dclenvvv}[4]{\constEnv, \typEnv \vdash {#1} \leadsto {#2} \dashv {#3}, {#4}}
\newcommand{\explCast}[2]{\castenv {e} {#1} {#2}}
\newcommand{\castenv}[3]{\env \vdash {#2} \rightarrow_{#1} {#3}}
\newcommand{\implCast}[2]{\castenv {i} {#1} {#2}}
%exp, args. exp', type pars, pars, kind, return type.
%check if you need the dir to be returned as well. 
\newcommand{\resolveFuncOver}[7]{\env, \ctxt \vdash \mathit{dispatch}({#1}, {#2}, {#3}, \funcType {#6} {#4} {#5} {#7})}
%typ par : typ arg. par = exp. . typ par : typ arg (infered).
% this has an extra input in code (which is constraints (init to typ par : typ arg)) but it's never used so i dropped it. same for return type.
\newcommand{\inferTypParArg}[3]{\env, \ctxt \vdash {#1}, {#2} \mapsto {#3}}
\newcommand{\unify}[6]{{#1}, {#2}, {#3} \vDash {#4} \Join {#5} \mapsto {#6}}
\newcommand{\resolveConstOver}[6]{\env, \ctxt \vdash \mathit{dispatch} \left( {#1}, {#2} \constructorTyp {#3} {#4} {#5} {#6}\right)}

%contexts:
% statement context
\newcommand{\expParserStateCtxt}{\mathsf{exp\_parser\_state}}
\newcommand{\expApplyBlockCtxt}{\mathsf{exp\_apply\_block}}
\newcommand{\expDclLocalCtxt}{\mathsf{exp\_dcl\_local}}
\newcommand{\expTableActionCtxt}{\mathsf{exp\_table\_action}}
\newcommand{\expActionCtxt}{\mathsf{exp\_action}}
\newcommand{\expMethodCtxt}{\mathsf{exp\_method}}
\newcommand{\expFuncCtxt}{\mathsf{exp\_function}}
\newcommand{\expConstantCtxt}{\mathsf{exp\_constant}}
% declaration context
\newcommand{\dclTopLevelCtxt}{\mathsf{dcl\_top\_level}}
\newcommand{\dclNestedCtxt}{\mathsf{dcl\_nested}}
\newcommand{\dclStmtCtxt}[1]{\mathsf{dcl\_stmt}{#1}}
%stmt context
\newcommand{\stmtCtxt}{\mathit{stmt\_ctxt}}
\newcommand{\stmtMethodCtxt}[1]{\mathsf{stmt\_method}\ {#1}}
\newcommand{\stmtFuncCtxt}[1]{\mathsf{stmt\_function}\ {#1}}
\newcommand{\stmtParserStateCtxt}{\mathsf{stmt\_parser\_state}}
\newcommand{\stmtApplyBlockCtxt}{\mathsf{stmt\_apply\_lock}}
\newcommand{\stmtActionCtxt}{\mathsf{stmt\_action}}
% parameter context
\newcommand{\runtimeCtxt}[1]{\mathsf{param\_runtime}\ \mathsf{#1}}
\newcommand{\constructorCtxt}[1]{\mathsf{param\_constructor}\ \mathsf{#1}}
%parser. control. method. action. function. package.
\newcommand{\ctxtPrm}{\mathit{ctxt\_prm}}
\newcommand{\parserCtxtPrm}{\mathsf{parser}}
\newcommand{\controlCtxtPrm}{\mathsf{control}}
\newcommand{\methodCtxtPrm}{\mathsf{method}}
\newcommand{\actionCtxtPrm}{\mathsf{action}}
\newcommand{\functionCtxtPrm}{\mathsf{function}}
\newcommand{\packageCtxtPrm}{\mathsf{package}}

%%keywords used in nontypename
\newcommand{\identifierkw}{\mathsf{identifier}}
\newcommand{\applykw}{\mathsf{apply}}
\newcommand{\keykw}{\mathsf{key}}
\newcommand{\actionskw}{\mathsf{actions}}
\newcommand{\statekw}{\mathsf{state}}
\newcommand{\entrieskw}{\mathsf{entries}}
\newcommand{\typekw}{\mathsf{type}}

%operators:
\newcommand{\mask}[2]{{#1}\ \&\&\&\ {#2}}
\newcommand{\range}[2]{{#1}..{#2}}
\newcommand{\ternary}[3]{{#1}\, ?\, {#2}\, : \, {#3}}
\newcommand{\errMem}[1]{\mathsf{error}.{#1}}
\newcommand{\typMem}[2]{{#1}.{#2}}
\newcommand{\shiftR}{>\!>}
\newcommand{\shiftL}{<\!<}
\newcommand{\leqOp}{<=}
\newcommand{\geqOp}{>=}
\newcommand{\concat}{+\!+}
\newcommand{\restOps}{\odot}
\newcommand{\ops}{$\odot$}
\newcommand{\unaryOp}{\ominus}
\newcommand{\binOp}{\oplus}
\newcommand{\cast}[2]{({#1})\,{#2}}
\newcommand{\plusSat}{|+|}
\newcommand{\subSat}{|-|}
\newcommand{\logor}{||}
\newcommand{\bitAnd}{\&}
\newcommand{\bitOr}{|}
\newcommand{\bitXor}{\string^}
\newcommand{\bitComplement}{\string~}
\renewcommand{\div}{/}
\renewcommand{\mod}{\%}
\newcommand{\funcCall}[3]{{#1}\, \langle{#2}\rangle\,({#3})}
\newcommand{\funcCallNoTypArgs}[2]{{#1}({#2})}
\newcommand{\expMem}[2]{{#1}.{#2}}
\newcommand{\instantiation}[2]{{#1}\, ({#2})} %%anonymous instantiation
%name.
%% \newcommand{\parserDecl}[]{\mathit{parser}\ {#1} (type_param param) {const_param locals states}
\renewcommand{\list}[1]{\{{#1}\}}
%% \newcommand{\record}[2]{\{\field_{#1} = \exp_{#1}, \ldots, \field_{#2} = \exp_{#2}\}}
\newcommand{\recordd}[3]{\{\field_{#1} = {#3}_{#1}, \ldots, \field_{#2} = {#3}_{#2}\}}
\newcommand{\record}[2]{\recordd {#1} {#2} {\exp} }
\newcommand{\records}[1]{\{\overline {\field = {#1}} \}}


%operators:
\newcommand{\intoOp}[1]{\texttt{#1}}

%metavariables:
\newcommand{\prmDef}{\mathit{prm}}
\newcommand{\prmDefs}{\overline {\prmDef}}
\newcommand{\constraint}{C}
\newcommand{\optConstraint}{B}
\newcommand{\nonoptConstraint}{N}
\newcommand{\typeEqs}{\Delta}
\newcommand{\anyTyp}{a}
\newcommand{\typEnv}{\mathit{typeEnv}}%{\Delta}
\newcommand{\typParEnv}{\mathit{typeParEnv}}%{\Upsilon}
\newcommand{\typSynEnv}{\mathit{typeSynEnv}}%{\Theta}
\newcommand{\varEnv}{\mathit{varEnv}}%{\Gamma}
\newcommand{\constEnv}{\mathit{constantEnv}}%{\Sigma}
\newcommand{\externEnv}{\mathit{externEnv}}%{\Xi}
\newcommand{\env}{\mathit{env}}%{\Omega}
\newcommand{\envDotEnv}[2]{{#1}.{#2}}
%% \newcommand{\typEnv}{\Gamma}
%% \newcommand{\constEnv}{\Sigma}
\newcommand{\typ}{\tau}
\newcommand{\typs}{\overline \typ}
\newcommand{\surfaceTyp}{\rho}
%% \newcommand{\surfaceTypOrVoidOrVar}{\rho_{+\voidTyp+\typVar}}
\newcommand{\surfaceTyps}{\overline {\surfaceTyp}}
\newcommand{\typArg}{\eta}
%% \newcommand{\funcProTyp}{\mu}
%% \newcommand{\funcProTyp}{\surfaceTypOrVoidOrVar}
\newcommand{\declaredTyp}{\nu}
\newcommand{\synTyp}{\sigma}
\newcommand{\typKind}{t}
\newcommand{\ctxt}{ctxt}
\newcommand{\ctxtTyp}{c}
\newcommand{\bool}{b}
\newcommand{\tru}{\mathsf{true}}
\newcommand{\fls}{\mathsf{false}}
\newcommand{\str}{s}
\newcommand{\width}{w}
\newcommand{\val}{v}
\newcommand{\bit}{b}
\renewcommand{\int}{n}
\newcommand{\bitWidth}[2]{{#2}\mathsf{w}{#1}}
\newcommand{\intWidth}[2]{{#2}\mathsf{s}{#1}}
\newcommand{\name}{x}
\newcommand{\var}{x}
\newcommand{\typVar}{X}
\newcommand{\typName}{t}
\newcommand{\typNames}{\overline \typName}
\newcommand{\typVars}{\overline \typVar}
\newcommand{\atyp}{at}
\newcommand{\atyps}{\overline \atyp}
\newcommand{\btyp}{bt}
\newcommand{\btyps}{\overline \btyp}
%% \renewcommand{\array}{a}
%% \renewcommand{\index}{i}
\newcommand{\arrayAccess}[2]{{#1}[{#2}]}
\newcommand{\size}{n}
\newcommand{\high}{h}
\newcommand{\low}{l}
\newcommand{\bitString}{bs}
\newcommand{\bitStringAccess}[3]{{#1}[{#2}:{#3}]}
\renewcommand{\exp}{\mathit{exp}}
\newcommand{\exps}{\overline {\exp}}
\newcommand{\field}{f}
\newcommand{\fields}{\overline \field}
\newcommand{\param}{x}
\newcommand{\params}{\overline x}
\newcommand{\prm}[1]{\dir_{#1}\,  \typ_{#1}\, \param_{#1}}
\newcommand{\singleprm}{\dir \, \typ \, \param }
\newcommand{\prms}{\overline {\singleprm}}
\newcommand{\singleprmm}{ {\VVal \dir \,  \VVal \typ \, \VVal \param}}
\newcommand{\typParam}{\typVar}
\newcommand{\typParams}{\overline \typParam}
\newcommand{\pari}[1]{\dir_{#1}\, \typ_{#1}\, \param_{#1} }
\newcommand{\paris}[1]{\overline {\pari {#1}}}
\newcommand{\pars}[2]{\pari {#1}, \ldots, \pari {#2}}
% wildcard param for wild card type vars.
\newcommand{\wildcard}{\_\!\_}
\newcommand{\wildcardParam}{W}
\newcommand{\wildcardParams}{\overline \wildcardParam}
\renewcommand{\arg}{\mathit{arg}}
\newcommand{\args}{\overline \arg}
\newcommand{\kind}{k}
\newcommand{\ret}{\mathit{return}}
\newcommand{\retTyp}{\typ_\ret}
\newcommand{\argexp}{\exp}
\newcommand{\argexps}{\overline \argexp}
\newcommand{\argexpi}[1]{\argexp_{#1}}
\newcommand{\argexpss}[2]{\argexpi {#1}, \ldots, \argexpi {#2}}
\newcommand{\argkv}{\var = \exp}
\newcommand{\argkvi}[1]{\var_{#1}=\exp_{#1}}
\newcommand{\argkvs}{\overline \argkv}
\newcommand{\argkvss}[2]{\argkvi {#1}, \ldots, \argkvi {#2}}
\newcommand{\missingarg}{\mathsf{missing}}
\newcommand{\names}{\overline \name}
\newcommand{\methods}[2]{\overline {{#1}:{#2}}}

% statements
\newcommand{\stmt}{\mathit{stmt}}
\newcommand{\stmts}{\overline \stmt}
\newcommand{\block}[1]{\{\overline {#1} \}}
\newcommand{\ifthen}[2]{\mathsf{if}\ {#1}\ \mathsf{then}\ {#2}}
\newcommand{\ifthenelse}[3]{\mathsf{if}\ {#1}\ \mathsf{then}\ {#2}\ \mathsf{else}\ {#3}}
\newcommand{\methodCall}[3]{{#1}\, \langle\overline {#2}\rangle\, \left(\overline {#3}\right)}
\newcommand{\assign}[2]{{#1}:={#2}}
\newcommand{\noop}{;}
\newcommand{\return}[1]{\mathsf{return}\ {#1}}
\newcommand{\retNothing}{\mathsf{return ;}}
\newcommand{\exit}{\mathsf{exit}}
\newcommand{\switch}[2]{\mathsf{switch}\ ({#1})\, \{\overline{#2}\}}
\newcommand{\lbl}{l}
\newcommand{\actionCase}[1]{\lbl : \block {#1}}
\newcommand{\switchCase}{\mathit{switch\_case}}
\newcommand{\switchCases}{\overline \switchCase}
\newcommand{\dirApp}[2]{{#1} \left(\overline{#2}\right)}
\newcommand{\defLbl}{\mathsf{default}}

%% parser states
\newcommand{\slctTran}[2]{({#1})\ \{{#2}\}}
\newcommand{\expandCase}[2]{{#1} \ {#2}}
\newcommand{\nxt}{\mathit{nxt}}
\newcommand{\accept}{\mathsf{accept}}
\newcommand{\reject}{\mathsf{reject}}
\newcommand{\dontcareMatch}{\mathsf{dontcare}}
\newcommand{\defMatch}{\mathsf{default}}
\newcommand{\pmatch}{\mathit{mtch}}
\newcommand{\pmatches}{\overline \pmatch}
\newcommand{\case}{\mathit{case}}
\newcommand{\pcases}{\overline \case}
\newcommand{\parserTran}{trans}
\newcommand{\parserTrans}{\overline \parserTran}
\newcommand{\stateDef}[3]{{#1}\ \{\overline{#2}\} \ {#3}}
\newcommand{\state}{\mathit{st}}
\newcommand{\states}{\overline \state}


%% table properties
\newcommand{\custom}[3]{{#1}\,({#2},{#3})}
\newcommand{\entry}[2]{\overline {#1} \, {#2}}
\newcommand{\action}{\mathit{act}}
\newcommand{\actionDef}[2]{{#1}\, ({#2})}
\newcommand{\actions}{\overline \action}
\newcommand{\key}[2]{{#1} : {#2}}
\newcommand{\keys}[2]{\overline {\key {#1} {#2}}}
\newcommand{\prop}{\mathit{prop}}
\newcommand{\props}{\overline \prop}

%% field type pairs
\newcommand{\fieldTyp}{\field:\surfaceTyp}
\newcommand{\fieldTyps}{\overline \fieldTyp}

%% methods
\newcommand{\constructor}[2]{{#1}\,(\overline {{#2}})}
\newcommand{\abstractMethod}[4]{{#1}\, {#2}\, \langle\overline {{#3}}\rangle\,(\overline {{#4}})}
\newcommand{\methodDef}[4]{{#1}\, {#2}\, \langle\overline {{#3}}\rangle\,(\overline {{#4}})}
\newcommand{\extMethod}{\mathit{mth}}
\newcommand{\extMethods}{\overline \extMethod}

% declarations
\newcommand{\dcl}{\mathit{dcl}}
\newcommand{\dcls}{\overline \dcl}
\newcommand{\constDcl}[3]{\mathsf{const}\ {#1}\ {#2} := {#3}}
\newcommand{\inst}[4]{{#1}\, (\overline{#2})\, {#3}\, \{\overline{#4}\}}
\newcommand{\instNoBlock}[3]{{#1}\, (\overline{#2})\, {#3}}
\newcommand{\parserDcl}[6]{\mathsf{parser}\ {#1}\, \langle \overline {#2}\rangle\, (\overline{#3})\, (\overline{#4})\, \{\overline {#5}\ \overline{#6} \}}
\newcommand{\controlDcl}[6]{\mathsf{control}\ {#1}\, \langle \overline {#2}\rangle\, (\overline{#3})\, (\overline{#4})\, \{ \overline{#5}\ \overline{#6} \}}
\newcommand{\funcDcl}[5]{\mathsf{function}\ {#1}\, {#2}\, \langle \overline {#3}\rangle\, (\overline{#4})\, \{ \overline{#5}\}}
\newcommand{\externFuncDcl}[4]{\mathsf{extern\_function}\ {#1}\, {#2}\, \langle \overline {#3}\rangle\, (\overline{#4})}
\newcommand{\varDclInit}[3]{{#1}\ {#2} := {#3}}
\newcommand{\varDcl}[2]{{#1}\, {#2}}
%type. expression. name.
\newcommand{\valSetDcl}[3]{{#1}\, {#3}:{#2}}
\newcommand{\actionDcl}[3]{\mathsf{action}\ {#1}\, (\overline{#2})\, \{\overline{#3}\}}
\newcommand{\tableDcl}[2]{\mathsf{table}\ {#1}\, \{\overline{#2}\}}
\newcommand{\headerDcl}[4]{\mathsf{header}\ {#1}\, \langle {\overline{#2}} \rangle\, \recordTypStructureClose{#3}{#4}}
\newcommand{\headerUnionDcl}[4]{\mathsf{header\_unsion}\ {#1}\, \langle {\overline{#2}} \rangle\, \recordTypStructureClose{#3}{#4}}
\newcommand{\structDcl}[4]{\mathsf{struct}\ {#1}\,  \langle {\overline{#2}} \rangle\, \recordTypStructureClose{#3}{#4}}
\newcommand{\errDcl}[1]{\mathsf{error}\ \{ \overline {#1} \}}
\newcommand{\matchkindDcl}[1]{\mathsf{match\_kind}\ \{\overline{#1}\}}
\newcommand{\enumDcl}[2]{\mathsf{enum}\ {#1}\, \{\overline{#2} \}}
\newcommand{\serEnumDcl}[4]{\mathsf{enum}\ {#1}\ {#2}\, \{\overline {{#3}={#4}} \}}
\newcommand{\externObjDcl}[3]{\mathsf{extern}\ {#1}\, \langle \overline {#2}\rangle\, \{\overline {#3} \}}
%type \typ \typename
\newcommand{\newtypeDcl}[2]{\mathsf{type} \ {#1}\ {#2}}
\newcommand{\typdefDcl}[2]{\mathsf{typedef}\ {#1}\ {#2}}
\newcommand{\controlTypDcl}[3]{\mathsf{control}\ {#1}\, \langle\overline{#2}\rangle\, (\overline{#3})}
\newcommand{\parserTypDcl}[3]{\mathsf{parser}\ {#1}\, \langle\overline{#2}\rangle\, (\overline{#3})}
\newcommand{\packageTypDcl}[3]{\mathsf{package}\ {#1}\, \langle\overline{#2}\rangle\, (\overline{#3})}

\newcommand{\prog}{\mathit{prog}}

%function helpers
\newcommand{\dclName}[1]{\mathit{name\_from\_dcl}({#1})}
\newcommand{\distinct}[1]{\mathit{are\_distinct}({#1})}
\newcommand{\noFreeTypVar}[1]{\mathit{no\_free\_type\_var}({#1})}
\newcommand{\types}[1]{\mathit{types\_of}({#1})}
\newcommand{\mergeConst}[1]{\mathit{merge\_types}({#1})}
\newcommand{\toVoid}[1]{\mathit{voidify}({#1})}
\newcommand{\breakMaybes}[3]{\mathit{break\_option}({#1})=\left({#2},{#3}\right)}
\newcommand{\isExtern}[1]{\mathit{is\_extern}({#1})}
\newcommand{\nothing}{\bot}
%% \newcommand{\maskTypeIs}[2]{\mathit{mask\_type}({#1},{#2})}
% \typ = \intWidthTyp \width \textOr \bitWidthTyp \width
\newcommand{\widthInt}[1]{\mathit{width\_int}({#1})}
\newcommand{\isWidthInt}[1]{\mathit{is\_width\_int}({#1})}
%% - $\bothWidthInt {\typ_1} {\typ_2}$ is a helper function that checks if both types are fixed-length signed or unsigned integers. That is, it checks the conditions below and returns $\tr$ if one of them holds:
%% + $\typ_1 = \typ_2 = \bitWidthTyp \width$
%% + $\typ_1 = \typ_2 = \intWidthTyp \width$
%% \newcommand{\bothWidthInt}[2]{\mathit{both\_width\_int}({#1},{#2})}
%\typ = {\bitWidthTyp \width} \textOr \intWidthTyp \width \textOr \integerTyp
\newcommand{\isInt}[1]{\mathit{is\_int}({#1})}
% \left( \typ_1 = \typ_2 = \integerTyp \right) \\
% \textOr \left( \prim {\typ_1} = \prim {\typ_2} = \bitWidthTyp \width \right) \\
% \textOr \left( \prim {\typ_1} = \prim {\typ_2} = \intWidthTyp \width \right)
\newcommand{\bothInt}[2]{\mathit{both\_int}({#1},{#2})}
\newcommand{\concatCond}[2]{\mathit{concat\_type}({#1},{#2})}
\newcommand{\divCond}[2]{\mathit{div\_type}({#1},{#2})}
\newcommand{\shiftCond}[2]{\mathit{shift\_condition}({#1},{#2})_{\env}}
\newcommand{\addTypeEnv}[2]{\typEnv[{#1}:{#2}]}
\newcommand{\addTypEnvv}[1]{\typEnv[{#1}]}
\newcommand{\unionEnv}[2]{{#1}\cup{#2}}
\newcommand{\addConstEnv}[2]{\constEnv[{#1}={#2}]}
\newcommand{\isNumeric}[2]{\mathit{is\_numeric}({#1}, {#2})}
\newcommand{\ifIntComp}[2]{\mathit{if\_int\_is\_compile\_time\_known}({#1}, {#2})}
\newcommand{\isNumOrEnum}[1]{\mathit{is\_num\_or\_enum}({#1})}
%% \newcommand{\isArray}[1]{\mathit{is\_array}({#1})}
\newcommand{\getEnum}[1]{\mathit{get\_enum}({#1})}
%% \newcommand{\isConstant}[1]{\mathit{is\_constant} ({#1})}
%% - $\isConstant \exp$ checks whether the expression $\exp$ is a constant or not.  
\newcommand{\isInstantiation}[1]{\mathit{is\_instantiation} ({#1})}
\newcommand{\isVariable}[1]{\mathit{is\_variable} ({#1})}
\newcommand{\compileTimeEval}[1]{\llbracket{#1}\rrbracket_{\env}}
\newcommand{\implicitCast}[2]{\mathit{implicit\_cast}({#1},{#2})}
\newcommand{\insertImplCast}[2]{\mathit{insert\_implicit\_cast}({#1},{#2})}
\newcommand{\insertEnumImplCast}[2]{\mathit{insert\_enum\_implicit\_cast}({#1},{#2})}
\newcommand{\reduceEnums}[1]{\mathit{reduce\_enums}({#1})_\env} 
\newcommand{\inOrLess}[2]{\mathit{in\_or\_directionless}({#1},{#2})}
%% \newcommand{\typEq}[3]{{#2} ==_{\env,{#1}} {#3}}
\newcommand{\typHasEq}[1]{\mathit{has\_equality}({#1})_\env}
\newcommand{\nonNeg}[1]{\mathit{is\_nonneg\_numeric}({#1})_\env}
\newcommand{\pos}[1]{\mathit{is\_pos\_numeric}({#1})_\env}
\newcommand{\compTimeKnown}[1]{\mathit{compile\_time\_known}({#1})_{\env}}
\newcommand{\sat}[1]{\mathit{saturate}({#1})_\typEnv}
\newcommand{\trans}[2]{\mathit{translate}({#1})_{\env,{#2}}}
\newcommand{\transMaybe}[1]{\mathit{optional\_translate}({#1})_{\env}}
\newcommand{\isValidNestedTyp}[1]{\mathit{is\_valid\_nested}({#1})_\env}
%{#1} is outer, 2 is inner
\newcommand{\noDup}[1]{\mathit{no\_duplicate}({#1})}
%% \newcommand{\explicitCastOK}[2]{\explCast {#1} {#2}}
\newcommand{\insertTypVars}[1]{\mathit{insert\_type\_vars}({#1}, \env)}
\newcommand{\getTypParams}[1]{\mathit{get\_type\_params}({#1})}
\newcommand{\getLength}[1]{|{#1}|}
\newcommand{\reduce}[1]{\mathit{reduce}({#1})_\typEnv}
\newcommand{\reduceWithEnv}[2]{\mathit{reduce}({#2})_{#1}} %%TODO: adjust env where it's used. 
%% \newcommand{\fieldOrMethodTyp}[1]{\mathit{field\_or\_method\_type}({#1})}
\newcommand{\isLval}[1]{\mathit{is\_lvalue}({#1})}
%type. exp. 
\newcommand{\castExpression}[2]{\mathit{cast\_expression}({#1},{#2})}
\newcommand{\ifVoidTyp}[1]{\mathit{if\_void\_type}({#1})}
\newcommand{\allowedTypeForVar}[1]{\mathit{is\_allowed\_type\_for\_variable}({#1})}
\newcommand{\checkParamShadow}[2]{\mathit{check\_parameter\_shadowing}({#1}, {#2})}
\newcommand{\match}[2]{\mathit{match}({#1},{#2})}
%% \newcommand{\zip}[2]{\mathit{zip} ({#1},{#2})}
\newcommand{\matchParArg}[2]{\mathit{macth\_parameter\_argument}({#1},{#2})}
\newcommand{\validatePars}[3]{\mathit{validate\_parameter}({#1})_{{#2},{#3}}}
\newcommand{\validateParArg}[1]{\mathit{validate\_parmeter\_argument}({#1})}
\newcommand{\maybe}[1]{{#1}_{\bot}}
\newcommand{\castParArg}[1]{\mathit{cast\_parameter\_argument}({#1})}
\newcommand{\callOK}[2]{\mathit{call\_ok}({#1},{#2})}
\newcommand{\removeOptionalPars}[1]{\mathit{remove\_optional\_parameteres}({#1})}
\newcommand{\concatList}[2]{{#1}\ +\!+\ {#2}}
\newcommand{\isDirectionless}[1]{\mathit{is\_directionless}({#1})}
\newcommand{\whatevs}{\_\!\_}
\newcommand{\valMatchTyp}[2]{\mathit{value\_matches\_type}( {#1}, {#2})}
\newcommand{\parTypEqInitTyp}[1]{\mathit{param\_type\_eq\_init\_type}({#1})}
\newcommand{\isSerializableTyp}[1]{\mathit{is\_serializable\_type}({#1})}
\newcommand{\getFieldTyp}[1]{\mathit{get\_field\_type}({#1})}
%fields 
\newcommand{\isValid}{\mathsf{isValid}()}
\newcommand{\apply}{\mathsf{apply}}
\newcommand{\sizingMem}{\mathit{sizeIn}}
\newcommand{\minSizeBits}{\mathsf{minSizeInBits}()}
\newcommand{\minSizeBytes}{\mathsf{minSizeInBytes}()}
\newcommand{\maxSizeBits}{\mathsf{maxSizeInBits}()}
\newcommand{\maxSizeBytes}{\mathsf{maxSizeInBytes}()}
\newcommand{\setValidity}{\mathit{setValidity}}
\newcommand{\setValid}{\mathsf{setValid}()}
\newcommand{\setInvalid}{\mathsf{setInvalid}()}
\newcommand{\pushFront}[1]{\mathsf{push\_front}({#1})}
\newcommand{\popFront}[1]{\mathsf{pop\_front}({#1})}
\newcommand{\pushOrPop}[1]{\mathit{push\_or\_pop}({#1})}
\newcommand{\sizeMem}{\mathsf{size}}
\newcommand{\nextIndex}{\mathsf{nextIndex}}
\newcommand{\lastIndex}{\mathsf{lastIndex}}
\newcommand{\nextOrLast}{\mathit{next\_or\_last}}
\renewcommand{\next}{\mathsf{next}}
\newcommand{\last}{\mathsf{last}}
\newcommand{\isHeader}[1]{\mathit{is\_header}({#1})}
\newcommand{\isHeaderOrUnion}[1]{\mathit{is\_header\_or\_union}({#1})}
\newcommand{\binOpExp}[3]{{#2}\, {#1}\, {#3}}


%kinds
\newcommand{\builtin}{\mathsf{builtin}}
\newcommand{\actionKind}{\mathsf{action}}
\newcommand{\parserKind}{\mathsf{parser}}
\newcommand{\controlKind}{\mathsf{control}}
\newcommand{\methodKind}{\mathsf{method}}
\newcommand{\externKind}{\mathsf{extern}}
\newcommand{\tableKind}{\mathsf{table}}
\newcommand{\funcKind}{\mathsf{function}}

%keywords
\newcommand{\newtypKey}{\mathsf{newtype}}

%values
\newcommand{\enumVal}[2]{\typMem {#1} {#2}}
\newcommand{\errVal}[1]{\errMem {#1}}
\newcommand{\matchkindVal}[1]{\mathsf{match\_kind}.{#1}}

%types:
\newcommand{\dontcareTyp}{\mathsf{dontcare}}
\newcommand{\notInstantiatedTyp}{\mathsf{not\_instantiated}}
\newcommand{\bitWidthTyp}[1]{\mathsf{bit}\langle{#1}\rangle}
\newcommand{\intWidthTyp}[1]{\mathsf{int}\langle{#1}\rangle}
\newcommand{\boolTyp}{\mathsf{bool}}
\newcommand{\stringTyp}{\mathsf{string}}
\newcommand{\integerTyp}{\mathsf{int}}
%% \newcommand{\bitStringTyp}[1]{\bitWidthTyp {#1}}
\newcommand{\setTyp}[1]{\mathsf{set}\langle{#1}\rangle}
\newcommand{\errTyp}{\mathsf{error}}
\newcommand{\varbitTyp}[1]{\mathsf{varbit}\langle{#1}\rangle}
\newcommand{\matchkindTyp}{\mathsf{match\_kind}}
\newcommand{\voidTyp}{\mathsf{void}}
\newcommand{\headerStackTyp}[2]{{#1}[{#2}]}
\newcommand{\tupleTyp}[2]{\mathsf{tuple}\langle{#1},\ldots,{#2}\rangle}
\newcommand{\tupleTyps}[1]{\mathsf{tuple}\langle\overline{#1}\rangle}
\newcommand{\listTyp}[2]{\tupleTyp {#1} {#2}}
\newcommand{\listTyps}[1]{\tupleTyps {#1}}
\newcommand{\listTypss}[1]{\{ {#1}\}}%for p4 workshop
\newcommand{\enumTypDef}{\mathsf{enum}\ {\typ}\ {\typName}\ \{\fields\}}%%TODO: add is surface typ to well-formed of this. 
\newcommand{\enumTyp}[3]{\mathsf{enum}\ {#1}\ {#2}\ \{{#3}\}}
\newcommand{\enumTypNoTypDef}{\mathsf{enum} \ {\typName}\ \{\fields\}}
\newcommand{\enumTypNoTyp}[2]{\mathsf{enum} \ {#1}\ \{{#2}\}}
%% \newcommand{\recordTypDef}{\recordTyp 1 n}
\newcommand{\recordTypStructureOpen}[2]{\{\field_{#1}:\typ_{#1};\ldots;\field_{#2}:\typ_{#2}\}}
\newcommand{\recordTypStructureClose}[2]{\{\overline {{#1} : {#2}}\}}
%% \newcommand{\recordTyp}[2]{\mathsf{record} \{\field_{#1}:\typ_{#1};\ldots;\field_{#2}:\typ_{#2}\}}
%% \newcommand{\recordTypp}[2]{\mathsf{record} \{\VVal {\field_{#1}}:\VVal {\typ_{#1}};\ldots;\VVal {\field_{#2}}:\VVal {\typ_{#2}}\}}
%% \newcommand{\recordTyps}{\mathsf{record} \recordTypStructureClose \field \typ}
%% \newcommand{\recordTyps}[2]{\mathsf{record} \recordTypStructureClose {#1} {#2}}
\newcommand{\headerUnionTyp}[2]{\mathsf{header\_union} \recordTypStructureOpen {#1} {#2}}
\newcommand{\headerUnionTyps}[3]{\mathsf{header\_union} \, \langle{\overline{#1}}\rangle \recordTypStructureClose {#2} {#3}}
\newcommand{\headerTyp}[3]{\mathsf{header}\, \langle{\overline{#1}}\rangle\recordTypStructureOpen {#2} {#3}}
\newcommand{\structTyp}[3]{\mathsf{struct}\, \langle{\overline{#1}}\rangle \recordTypStructureOpen {#2} {#3}}
\newcommand{\headerTyps}[3]{\mathsf{header}\, \langle{\overline{#1}}\rangle\recordTypStructureClose {#2} {#3}}
%% \newcommand{\headerTypsNoGen}[2]{\mathsf{header} \recordTypStructureClose {#1} {#2}}
%% \newcommand{\headerTypp}[2]{\mathsf{header} \recordTypp {#1} {#2}}
%% \newcommand{\structTypp}[2]{\mathsf{struct} \recordTypp {#1} {#2}}
\newcommand{\structTypDef}{\mathsf{struct} \recordTypStructureOpen 1 n}
\newcommand{\structTyps}[3]{\mathsf{struct}\, \langle{\overline{#1}}\rangle \recordTypStructureClose {#2} {#3}}
%% \newcommand{\structTypsNoGen}[2]{\mathsf{struct} \recordTypStructureClose {#1} {#2}}
%% \newcommand{\headerUnionTyp}{\mathit{header\_union}\{\field_1:\typ_1;\ldots;\field_n:\typ_n \}}
%% \newcommand{\headerTypDef}{\headerTyp 1 n}
%% \newcommand{\headerTyp}[2]{\mathit{header}\{\field_{#1}:\typ_{#1};\ldots;\field_{#2}:\typ_{#2}\}}
%% \newcommand{\headerTypp}[2]{\mathit{header}\{\VVal {\field_{#1}}:\VVal {\typ_{#1}};\ldots;\VVal {\field_{#2}}:\VVal {\typ_{#2}}\}}
%% \newcommand{\structTypDef}{\structTyp 1 n}
%% \newcommand{\structTyp}[2]{\mathit{struct}\{\field_{#1}:\typ_{#1};\ldots;\field_{#2}:\typ_{#2}\}}
%% \newcommand{\structTypp}[2]{\mathit{struct}\{\VVal {\field_{#1}}:\VVal {\typ_{#1}};\ldots;\VVal {\field_{#2}}:\VVal {\typ_{#2}}\}}
\newcommand{\newTypeTyp}[2]{\newtypKey \ {#1}\ {#2}}
%% \newcommand{\funcTyp}[3]{\mathsf{function}\ {#1}\ \langle{#2}\rangle ({#3})}
\newcommand{\simpFuncTyp}[2]{{#1} \rightarrow {#2}}
%kind.type param. param. return type. 
\newcommand{\funcType}[4]{\mathsf{function}\ {#1}\ \langle\overline{#2}\rangle (\overline{#3})\rightarrow {#4}}
% type params: list of name. wildcard param: list of name. params: list of type dir name. return: typed.
%% second arg is wildcard which is omitted for now. 
%% \newcommand{\constructorTyp}[4]{\mathsf{constructor}\langle{#1},{#2}\rangle\ ({#3}) \rightarrow {#4}}
\newcommand{\constructorTyp}[3]{\mathsf{constructor}\, \langle\overline{#1}\rangle (\overline{#2}) \rightarrow {#3}}
\newcommand{\externTyp}[1]{\mathsf{extern} \  {#1}}
\newcommand{\spcTyp}[2]{{#1}\langle\overline{#2}\rangle}
\newcommand{\parserTyp}[2]{\mathsf{parser}\, \langle\overline{#1}\rangle(\overline{#2})}
\newcommand{\controlTyp}[2]{\mathsf{control}\, \langle\overline{#1}\rangle(\overline{#2})}
\newcommand{\packageTyp}[2]{\mathsf{package}\, \langle\overline{#1}\rangle(\overline{#2})}
%% the third arg is wildcard which is omitted for now.
%% \newcommand{\packageTyp}[3]{\mathsf{package}\langle{#1}\rangle({#2},{#3})}
\newcommand{\actionTyp}[2]{\mathsf{action}(\mathsf{data:}\ \overline{#1},\mathsf{control:}\ \overline{#2})}
\newcommand{\tableTyp}[1]{\mathsf{table}\ {#1}}
%% \newcommand{\typNameTyp}[1]{\mathsf{type\_name}\ {#1}}
\newcommand{\unitTyp}{()}
%% \newcommand{\stackTyp}[2]{{#1}[{#2}]}

%directions:
\newcommand{\less}{\mathsf{directionless}}
\newcommand{\dir}{d}
\newcommand{\direction}{\mathit{dir}} %%for the typed.direction
\newcommand{\dirs}{\overline \dir}
\newcommand{\inDir}{\mathsf{in}}
\newcommand{\out}{\mathsf{out}}
\newcommand{\inout}{\mathsf{in\_out}}


%externs:
\newcommand{\externmethod}{(\name, \typ)}


\newcommand{\with}[2]{{#1}\ \mathsf{with} \ {#2}}

%helpers for types
\newcommand{\isSurface}[1]{\mathit{is\_surface}({#1})}

%helpers for packet types
\newcommand{\hasPacketIn}[1]{\mathit{has\_packet\_in}({#1})}
\newcommand{\ifTypeExternDirLess}[2]{\mathit{if\_type\_extern\_dir\_less}(\overline {{#1}\ {#2}})}
\newcommand{\packetIn}{\mathsf{packet\_in}}
\newcommand{\dropInit}[1]{\mathit{drop\_initializer}({#1})}
\newcommand{\validParam}[3]{\mathit{valid\_param}({#1}, {#2})_{#3}}
